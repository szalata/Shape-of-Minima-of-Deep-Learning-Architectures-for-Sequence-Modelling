\documentclass[10pt,conference,compsocconf]{IEEEtran}

% Packages
\usepackage{amsfonts}
\usepackage{amsmath}
\usepackage{bbm}
\usepackage{cite}
\usepackage{hyperref}
	\def\sectionautorefname{Section}
	\def\subsectionautorefname{Section}
	\def\figureautorefname{Fig.}
\usepackage{graphicx}
	\setkeys{Gin}{width=.5\textwidth, totalheight=\textheight, keepaspectratio}
	\graphicspath{{graphics/}}
\usepackage{xspace}

% Formatting
\newcommand{\parabf}[1]{\vspace{1mm}\noindent\textbf{#1}}
\newcommand{\parait}[1]{\vspace{1mm}\noindent\textit{#1}}

% Macros
\newcommand{\Mtrain}{\mathtt{M}^{\mathtt{train}}}
\newcommand{\Mtest}{\mathtt{M}^{\mathtt{test}}}
\newcommand{\Strain}{\mathtt{S}^{\mathtt{train}}}
\newcommand{\Stest}{\mathtt{S}^{\mathtt{test}}}
\newcommand{\fixed}{\mathtt{fixed}}
\newcommand{\rot}{\mathtt{rot}}
\newcommand{\plain}{\texttt{plain}\xspace}
\newcommand{\manual}{\texttt{plain\_manual}\xspace}
\newcommand{\sensor}{\texttt{plain\_sensor}\xspace}
\newcommand{\cipher}{\texttt{cipher}\xspace}
\newcommand{\msg}{\mathbf{m}}
\newcommand{\cnn}{\texttt{CharCNN}\xspace}


\begin{document}
\title{Shape of minima}

\author{
  
Sepideh Mamooler, Artur Szalata, Mattia Mariantoni\\
  \textit{EPFL, Switzerland}
}

\maketitle

\begin{abstract}
In this project we explore the relationship between a neural network to generalize and various metrics for sharpness of its minima based on the hessian of the loss function. The model we considered are transformers and LSTM.
\end{abstract}

%%%%%%%%%%%%%%%%%%%%%%%%%%%%%%%%
%%%%%%%%%%% Introduction %%%%%%%
%%%%%%%%%%%%%%%%%%%%%%%%%%%%%%%%
\section{Introduction}\label{sec:intro}
Something \cite{keskar}

%%%%%%%%%%%%%%%%%%%%%%%%%%%%%%%%
%%%%%%%%%%% Background %%%%%%%
%%%%%%%%%%%%%%%%%%%%%%%%%%%%%%%%
\section{Experiments}\label{sec:data}


\section{Methods}\label{sec:model}
\parabf{Sharpness.} We have considered various metrics based on norm the norm of the gradient or hessian of the loss function. The only gradient based metric we used takes into consideration the norm-2 of the gradient of the loss function. The other metrics are all hessian based and some analyze characteristics of the eigenvalues of the hessian such as max or mean eigenvalues, the other are based on the norm of the hessian, both the norm-2 and nuclear norm, while the remaining are based on the trace and determinant of the hessian.

%%%%%%%%%%%%%%%%%%%%%%%%%%%%%%%%
%%%%%%%%%%% Results %%%%%%%
%%%%%%%%%%%%%%%%%%%%%%%%%%%%%%%%
\section{Results and conclusions}\label{sec:results}




\section*{Acknowledgments}
We would like to thank et al...

\bibliography{references}
\bibliographystyle{IEEEtran}


\end{document}
